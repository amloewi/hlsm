\documentclass[11pt, oneside]{article}   	% use "amsart" instead of "article" for AMSLaTeX format
\usepackage{geometry}                		% See geometry.pdf to learn the layout options. There are lots.
\geometry{letterpaper}                   		% ... or a4paper or a5paper or ... 
%\geometry{landscape}                		% Activate for rotated page geometry
%\usepackage[parfill]{parskip}    		% Activate to begin paragraphs with an empty line rather than an indent
\usepackage{graphicx}				% Use pdf, png, jpg, or eps§ with pdflatex; use eps in DVI mode
								% TeX will automatically convert eps --> pdf in pdflatex		
\usepackage{amssymb, amsmath}

%SetFonts

%SetFonts

\begin{document}

The Binomial likelihood equation of the HLSM is:

\begin{align*}
  L &= \prod_k \prod_{i}\prod_{j<i} \sigma(\eta_{ijk})^{y_{ijk}}(1-\sigma(\eta_{ijk}))^{1-y_{ijk}}
\end{align*}

where,

\[
\eta_{ijk} = \alpha_k - \|z_{ik} - z_{jk}\|^2 =  \alpha_k - \|b_i + \epsilon_{ik} - b_j - \epsilon_{jk}\|^2
\]

\[
z_{ik} = b_i + \epsilon_{ik}
\]

\[
\sigma(\eta_{ijk}) = \frac{1}{1+\exp(-\eta_{ijk})}
\]

Then, the log likelihood is

\begin{align*}
  \ell 	&= \sum_k \sum_{i} \sum_{j < i} {y_{ijk}} \ln \sigma(\eta_{ijk}) + (1-y_{ijk})\ln(1- \sigma(\eta_{ijk})) \\
  	&= \sum_k \sum_{i} \sum_{j < i} - {y_{ijk}} \ln (1+\exp(-\eta_{ijk})) + (1-y_{ijk})\ln\left (\frac{\exp(-\eta_{ijk})}{1+\exp(-\eta_{ijk})}\right) \\
  	&= \sum_k \sum_{i} \sum_{j < i} - {y_{ijk}} \ln (1+\exp(-\eta_{ijk})) + (1-y_{ijk})[-\eta_{ijk} - \ln (1+\exp(-\eta_{ijk}))] \\
  	&=\sum_k \sum_{i} \sum_{j < i} - {y_{ijk}} \ln (1+\exp(-\eta_{ijk})) -\eta_{ijk} - \ln (1+\exp(-\eta_{ijk})) + y_{ijk} \eta_{ijk} + y_{ijk}\ln (1+\exp(-\eta_{ijk}))] \\
 	& = \sum_k \sum_{i} \sum_{j < i} (y_{ijk}-1)\eta_{ijk} - \ln (1+\exp(-\eta_{ijk}))
\end{align*}

We define our decision variables as: $\{b_i\}$ $\forall~i$ and $\{\epsilon_{ik}\}$ $\forall~i,k$.

Then, the gradients of the $\eta_{ijk}$ are:

\[
\frac{\partial \eta_{ijk}}{\partial b_{i}} =  - 2(b_i + \epsilon_{ik} - b_j - \epsilon_{jk})
\]

\[
\frac{\partial \eta_{ijk}}{\partial b_{j}} = 2(b_i + \epsilon_{ik} - b_j - \epsilon_{jk})
\]

\[
\frac{\partial \eta_{ijk}}{\partial \epsilon_{ik}} = - 2(b_i + \epsilon_{ik} - b_j - \epsilon_{jk})
\]

\[
\frac{\partial \eta_{ijk}}{\partial \epsilon_{jk}} = 2(b_i + \epsilon_{ik} - b_j - \epsilon_{jk})
\]

\begin{align*}
\frac{\partial \ell}{\partial b_{m}} & =  \sum_k \sum_{i}  \sum_{j<i} (y_{ijk} -1)\frac{\partial \eta_{ijk}}{\partial b_{m}} - \frac{\partial \ln (1+\exp(-\eta_{ijk}))}{\partial b_{m}} \\
& =  \sum_k\left[ \sum_{j<m} (y_{mjk} -1)\frac{\partial \eta_{mjk}}{\partial b_{m}} - \frac{\partial \ln (1+\exp(-\eta_{mjk}))}{\partial b_{m}} +  \sum_{i > m} (y_{imk} -1)\frac{\partial \eta_{imk}}{\partial b_{m}} - \frac{\partial \ln (1+\exp(-\eta_{imk}))}{\partial b_{m}} \right] \\
%
& =  \sum_k\bigg[ -2 \sum_{j<m} \left[(y_{mjk} -1)(b_m + \epsilon_{mk} - b_j + \epsilon_{jk}) + \frac{\exp(-\eta_{mjk})}{1+\exp(-\eta_{mjk})}(b_m + \epsilon_{mk} - b_j - \epsilon_{jk}) \right] \\
& \quad\quad\quad +  2 \sum_{i > m} \left[(y_{imk} -1)(b_i + \epsilon_{ik} - b_m - \epsilon_{mk}) +\frac{\exp(-\eta_{ijk})}{1+\exp(-\eta_{mjk})}(b_i + \epsilon_{ik} - b_m - \epsilon_{mk}) \right] \bigg] \\
& =  2 \sum_k\bigg[ - \sum_{j<m} \left(y_{mjk} -1 + \frac{\exp(-\eta_{mjk})}{1+\exp(-\eta_{mjk})}\right)(z_{mk} - z_{jk})\\
& \quad\quad\quad\quad +  \sum_{i > m} \left(y_{imk} -1 + \frac{\exp(-\eta_{ijk})}{1+\exp(-\eta_{mjk})}\right)(z_{ik} - z_{mk}) \bigg] \\
%
\end{align*}

\begin{align*}
\frac{\partial \ell}{\partial \epsilon_{mk}} & = 2 \bigg[ - \sum_{j<m} \left(y_{mjk} -1 + \frac{\exp(-\eta_{mjk})}{1+\exp(-\eta_{mjk})}\right)(z_{mk} - z_{jk})\\
& \quad\quad +  \sum_{i > m} \left(y_{imk} -1 + \frac{\exp(-\eta_{ijk})}{1+\exp(-\eta_{mjk})}\right)(z_{ik} - z_{mk}) \bigg]
\end{align*}


Finally, our complete objective function is to minimize the sum between the negative log likelihood and a lasso penalty on the deviations $\epsilon_{ik}$.

\begin{align*}
\min_{\epsilon, b} \sum_k \sum_{i} \sum_{j < i} [ (1 - y_{ijk})\eta_{ijk} +  \ln (1+\exp(\eta_{ijk})) ] + \lambda \sum_k \sum_i \| \epsilon_{i,k}\|
\end{align*}

where $\|.\|$ represents the L2 norm. Since this is not differentiable, we could use proximal gradient descent. If the gradients for the ML function derived above are correct, we can directly use the code we implemented in HW3 (problem 4) to implement the proximal operator for this problem.

The proximal operator for the penalty term using the L2 norm\footnotemark~ is:

\footnotetext{According to the results on Homework 3}

\[
\text{prox}_{\|.\|, t}(\epsilon_{i,k}) = 
\begin{cases}
\frac{\|\epsilon_{i,k}\| - \lambda t}{\|\epsilon_{i,k}\|}\epsilon_{i,k}, & \|\epsilon_{i,k}\| \ge \lambda t \\
0, & \|\epsilon_{i,k}\| < \lambda t
\end{cases}
\]

\[
\text{prox}_{\|.\|, t}(b_i) = b_i
\]

Therefore, let $\beta$ be the vector with all parameters $\epsilon$ and $b$. The Proximal Gradient Descent method in this case is:

\[
\beta^{+} = \text{prox}_{\|.\|, t}(\beta - t \nabla \ell(\beta))
\]

\end{document}  